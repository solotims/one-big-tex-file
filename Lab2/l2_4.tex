\section{Цель работы\\}
Ознакомиться с методами идентификации характеристик системы и расчета коэффициентов регуляторов.

\section{Порядок выполнения работы\\}
\begin{enumerate} 
\item[1.] Ознакомиться с теоретической частью лабораторной работы.
\item[2.] Рассчитать параметры $J_m, K_e, K_m$ сочленения по формулам \eqref{coe}.
\item[3.] Подавая на сочленение напряжение  $u(t) = U_m\sin{{\omega}t}$, при трех разных $\omega$ снять показания углов поворота и соответствующих значений времени.
\item[4.] Аппроксимировать с помощью \textit{datafit} полученные данные формулой \eqref{theta} по $A$, получить графики, подобные изображенному на рис.~\ref{yesfr}
\item[5.] Вычислить коэффициент трения $K_f$ для каждой $\omega$. Итоговое значение $K_f$ рассчитать как среднее арифметическое. 
\item[6.] Рассчитать коэффициенты ПИД-регулятора угла поворота сочленения подстановкой полиномов Ньютона и Баттерворта для нескольких $t_p$. Провести эксперименты, применив anti-windup, и выбрать такое $t_p$, при котором колебания минимальны. Необходимо учесть, что согласно рис.~\ref{pid} ПИД-регулятор рассчитывает момент напряжения, а на вход EV3 подается напряжение. Соответственно, рассчитанный регулятором сигнал нужно сначала перевести сначала в вольты, домножив на $\frac{R}{K_m}$, а затем в проценты.
\item[7.] Сделать моделирование в \textit{xcos}. Построить графики, аналогичные рис.~\ref{pidmod} и ~\ref{pidmod2} .
\item[8.] Рассчитать коэффициенты ПИ-регулятора угловой скорости сочленения подстановкой полиномов Ньютона и Баттерворта для нескольких $t_p$. Провести эксперименты и выбрать такое $t_p$, при которых скорость наиболее стабильна.
\item[9.] Сделать моделирование в \textit{xcos}. Построить графики, аналогичные рис.~\ref{pimod} и ~\ref{pimod2} .
 \end{enumerate}
 

\section{Содержание отчета\\}
\begin{enumerate} 
\item[1.] Указание найденного методом наименьших квадратов коэффициента трения $K_f$.
\item[2.] 3 графика с аппроксимациями.
\item[3.] Графики переходных пNроцессов регуляторов угла поворота сочленения.
\item[4.] Графики переходных процессов регуляторов угловой скорости сочленения.
\item[5.] Вывод о результате проделанной работе.
 \end{enumerate}
 
\begin{figure}[h]
	\noindent\centering{\includegraphics[height = 6cm]{img/N1.png}}
	\caption{График ПИД-регулятора с коэффициентами, полученными из бинома Ньютона при времени переходного процесса 0.85 сек и anti-windup 0.1}
	\label{pidmod}
\end{figure}

\begin{figure}[h]
	\noindent\centering{\includegraphics[height = 6cm]{img/B1.png}}
	\caption{График ПИД-регулятора с коэффициентами, полученными из полинома Баттерворта при времени переходного процесса 0.54 сек и anti-windup 0.3}
	\label{pidmod2}
\end{figure}

\begin{figure}[h]
	\noindent\centering{\includegraphics[height = 6cm]{img/N2.png}}
	\caption{График ПИ-регулятора с коэффициентами, полученными из бинома Ньютона при времени переходного процесса 0.43 сек }
	\label{pimod}
\end{figure}

\begin{figure}[h]
	\noindent\centering{\includegraphics[height = 6cm]{img/B2.png}}
	\caption{График ПИ-регулятора с коэффициентами, полученными из полинома Баттерворта при времени переходного процесса 0.18 сек }
	\label{pimod2}
\end{figure}
